%! Author = Christoph Renzing
%! Date = 15.08.2022

% Preamble
\documentclass[11pt]{PyRollDocs}
\usepackage{mathtools}
\addbibresource{refs.bib}

% Document
\begin{document}

    \title{The Lippmann - Mahrenholz Power and Labour PyRoll Plugin}
    \author{Christoph Renzing}
    \date{\today}

    \maketitle

    This plugin provides the roll force and roll torque model developed by \textcite{Lippmann_Mahrenholtz_1967}.
    Tge method was inspired by the solution of \textcite{Sims1954}.
    The basic equations are derived from classic strip theory with suitable simplifications suitable for hot rolling.
    For the presented plugin, the specific roll torque at the upper work roll is calculated.
    Usage of the equations for groove rolling is only valid, when using a equivalent rectangle approach.
    Heights used for calculation are therefore equivalent values for a equivalent flat pass.
    Furthermore, the used variable $h$ is the height of the equivalent flat workpiece and $b_m$ it's mean width,
    $k_{f,m}$ represents the mean flow stress of the material and $\epsilon$ the reduction of the equivalent pass in height direction.
    The indices 0 and 1 denote the incoming and exiting profile and $L_d$ is the contact length of the pass.


    \section{Model approach}\label{sec:model-approach}

    To calculate the roll force in hot rolling, the following equation was developed:

    \begin{equation}
        F_{Roll} = A_d k_{f,m} Q_F
        \label{eq:lippmann-mahrenholz-force}
    \end{equation}

    The function $Q_{Force}$ is the inverse forming efficiency and is calculated using equation~\eqref{eq:inverse-forming-efficiency}.
    This function depends on the neutral line angle $\beta_n$, which is calculated through equation~\eqref{eq:neutral-line-angle}.
    $\sigma_R$ and $\sigma_V$ are the forward and backward tension applied to the pass.

    \begin{subequations}
        \begin{equation}
            \begin{multlined}
                Q_{F} = \frac{\sigma_R}{k_{f,m}} + 2 \sqrt{\frac{1 - \epsilon}{\epsilon}} \arctan\left( \sqrt{\frac{\epsilon}{1 - \epsilon}} \right) - 1 +
                \sqrt{\frac{R}{h_1}} \sqrt{\frac{1 - \epsilon}{\epsilon}}  \log\left( \frac{\sqrt{1 - \epsilon}}{1 - \epsilon (1 - \beta_n ^2)} \right) \\
            \end{multlined}
            \label{eq:inverse-forming-efficiency}
        \end{equation}
        \begin{equation}
            \beta_n = \sqrt{\frac{1 - \epsilon}{\epsilon}} \tan \left(  \frac{1}{2} \sqrt{\frac{h_1}{R}} \left[ \frac{\sigma_R - \sigma_V}{k_{f,m}} + \log (1 - \epsilon) \right] + \frac{1}{2} \arctan \sqrt{\frac{\epsilon}{1 - \epsilon}}\right)
            \label{eq:neutral-line-angle}
        \end{equation}
        \begin{equation}
            k_{f,m} = \frac{k_{f, 0} + 2 k_{f, 1}}{3}
            \label{eq:mean-flow-stress}
        \end{equation}
    \end{subequations}

    As for the roll torque $M_{roll}$ at a single roll, Lippmann and Mahrenholz developed a similar equation.
    \begin{subequations}
        \begin{equation}
            M_{roll} = b_m R k_{f, m} Q_M \Delta h
            \label{eq:lippmann-mahrenholz-torque}
        \end{equation}
        \begin{equation}
            Q_M = \sqrt{\frac{R}{h_1}}\sqrt{\frac{1 - \epsilon}{\epsilon}}\left( \frac{1}{2} - beta_n \right)
            \label{eq:lippmann-mahrenholz-torque-efficiency}
        \end{equation}
    \end{subequations}


    \section{Usage instructions}\label{sec:usage-instructions}

    The plugin can be loaded under the name \texttt{pyroll\_lippmann\_mahrenholz\_\_power\_and\_labour}.

    An implementation of the \lstinline{roll_force} and \lstinline{roll_torque} hook on \lstinline{RollPass} and \lstinline{RollPass.Roll} is provided,
    calculating the roll force and torque using~\eqref{eq:lippmann-mahrenholz-torque} and~\eqref{eq:lippmann-mahrenholz-force}.
    Several additional hooks on \lstinline{RollPass} are defined, which are used in power and labour calculations, as listed in \autoref{tab:hookspecs}.
    Base implementations of them are provided, so it should work out of the box.
    Provide your own hook implementations or set attributes on the \lstinline{RollPass} instances to alter the spreading behavior.

    \begin{table}
        \centering
        \caption{Hooks specified by this plugin.}
        \label{tab:hookspecs}
        \begin{tabular}{ll}
            \toprule
            Hook name                                             & Meaning                                               \\
            \midrule
            \texttt{equivalent\_reduction}                        & Reduction $\epsilon$                                  \\
            \texttt{neutral\_line\_angle}             & Angle of neutral line $\alpha_n$                      \\
            \bottomrule
        \end{tabular}
    \end{table}

    \printbibliography


\end{document}