%! Author = Christoph Renzing
%! Date = 15.08.2022

% Preamble
\documentclass[11pt]{PyRollDocs}
\usepackage{mathtools}
\addbibresource{refs.bib}

% Document
\begin{document}

    \title{The Lippmann-Mahrenholz Force and Torque PyRolL Plugin}
    \author{Christoph Renzing}
    \date{\today}

    \maketitle

    This plugin provides the roll force and roll torque model developed by \textcite{Lippmann_Mahrenholtz_1967}.
    The basic equations are derived from classic strip theory with simplifications suitable for hot rolling.
    For the presented plugin, the specific roll torque at the upper work roll is calculated.
    Usage of the equations for groove rolling is only valid, when using an equivalent rectangle approach.
    Furthermore, the used variable $h$ is the height of the equivalent flat workpiece and $b_m$ it's mean width,
    $k_{f,m}$ represents the mean flow stress of the material and $\epsilon$ the relative drought of the pass.
    The indices 0 and 1 denote the incoming and exiting states and $L_d$ is the contact length of the pass.


    \section{Model approach}\label{sec:model-approach}

    To calculate the roll force in hot rolling, the following equation was developed:

    \begin{equation}
        F_{Roll} = A_d k_{f,m} Q_F
        \label{eq:lippmann-mahrenholz-force}
    \end{equation}

    The function $Q_{Force}$ is the inverse forming efficiency and is calculated using equation~\eqref{eq:inverse-forming-efficiency}.
    This function depends on the neutral line angle $\beta_n$, which is calculated through equation~\eqref{eq:neutral-line-angle}.
    $\sigma_R$ and $\sigma_V$ are the forward and backward tension applied to the pass.

    \begin{subequations}
        \begin{equation}
            \begin{multlined}
                Q_{F} = \frac{\sigma_R}{k_{f,m}} + 2 \sqrt{\frac{1 - \abs{\epsilon}}{\abs{\epsilon}}} \arctan\left( \sqrt{\frac{\abs{\epsilon}}{1 - \abs{\epsilon}}} \right) - 1 +
                \sqrt{\frac{R}{h_1}} \sqrt{\frac{1 - \abs{\epsilon}}{\abs{\epsilon}}}  \log\left( \frac{\sqrt{1 - \abs{\epsilon}}}{1 - \abs{\epsilon} (1 - \beta_n ^2)} \right) \\
            \end{multlined}
            \label{eq:inverse-forming-efficiency}
        \end{equation}
        \begin{equation}
            \beta_n = \sqrt{\frac{1 - \abs{\epsilon}}{\abs{\epsilon}}} \tan \left(  \frac{1}{2} \sqrt{\frac{h_1}{R}} \left[ \frac{\sigma_R - \sigma_V}{k_{f,m}} + \log (1 - \abs{\epsilon}) \right] + \frac{1}{2} \arctan \sqrt{\frac{\abs{\epsilon}}{1 - \abs{\epsilon}}}\right)
            \label{eq:neutral-line-angle}
        \end{equation}
        \begin{equation}
            k_{f,m} = \frac{k_{f, 0} + 2 k_{f, 1}}{3}
            \label{eq:mean-flow-stress}
        \end{equation}
    \end{subequations}

    As for the roll torque $M_{roll}$ at a single roll, Lippmann and Mahrenholz developed a similar equation.
    \begin{subequations}
        \begin{equation}
            M_{roll} = b_m R k_{f, m} Q_M \Delta h
            \label{eq:lippmann-mahrenholz-torque}
        \end{equation}
        \begin{equation}
            Q_M = \sqrt{\frac{R}{h_1}}\sqrt{\frac{1 - \abs{\epsilon}}{\abs{\epsilon}}}\left( \frac{1}{2} - \beta_n \right)
            \label{eq:lippmann-mahrenholz-torque-efficiency}
        \end{equation}
    \end{subequations}


    \section{Usage instructions}\label{sec:usage-instructions}

    The plugin can be loaded under the name \texttt{pyroll\_lippmann\_mahrenholz\_force\_torque}.

    An implementation of the \lstinline{roll_force} and \lstinline{roll_torque} hook on \lstinline{RollPass} and \lstinline{RollPass.Roll} is provided,
    calculating the roll force and torque using~\eqref{eq:lippmann-mahrenholz-force} and~\eqref{eq:lippmann-mahrenholz-torque}.
    Values for the \lstinline{mean_front_tension} as well as \lstinline{mean_back_tension} have to be provided by the user for the \lstinline{RollPass}.
    For the hook \lstinline{mean_neutral_line_angle} on \lstinline{RollPass} \autoref{eq:neutral-line-angle} is implemented.

    \printbibliography


\end{document}